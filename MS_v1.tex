\documentclass[a4paper,10pt]{article}\usepackage[]{graphicx}\usepackage[]{color}
%% maxwidth is the original width if it is less than linewidth
%% otherwise use linewidth (to make sure the graphics do not exceed the margin)
\makeatletter
\def\maxwidth{ %
  \ifdim\Gin@nat@width>\linewidth
    \linewidth
  \else
    \Gin@nat@width
  \fi
}
\makeatother

\definecolor{fgcolor}{rgb}{0.345, 0.345, 0.345}
\newcommand{\hlnum}[1]{\textcolor[rgb]{0.686,0.059,0.569}{#1}}%
\newcommand{\hlstr}[1]{\textcolor[rgb]{0.192,0.494,0.8}{#1}}%
\newcommand{\hlcom}[1]{\textcolor[rgb]{0.678,0.584,0.686}{\textit{#1}}}%
\newcommand{\hlopt}[1]{\textcolor[rgb]{0,0,0}{#1}}%
\newcommand{\hlstd}[1]{\textcolor[rgb]{0.345,0.345,0.345}{#1}}%
\newcommand{\hlkwa}[1]{\textcolor[rgb]{0.161,0.373,0.58}{\textbf{#1}}}%
\newcommand{\hlkwb}[1]{\textcolor[rgb]{0.69,0.353,0.396}{#1}}%
\newcommand{\hlkwc}[1]{\textcolor[rgb]{0.333,0.667,0.333}{#1}}%
\newcommand{\hlkwd}[1]{\textcolor[rgb]{0.737,0.353,0.396}{\textbf{#1}}}%

\usepackage{framed}
\makeatletter
\newenvironment{kframe}{%
 \def\at@end@of@kframe{}%
 \ifinner\ifhmode%
  \def\at@end@of@kframe{\end{minipage}}%
  \begin{minipage}{\columnwidth}%
 \fi\fi%
 \def\FrameCommand##1{\hskip\@totalleftmargin \hskip-\fboxsep
 \colorbox{shadecolor}{##1}\hskip-\fboxsep
     % There is no \\@totalrightmargin, so:
     \hskip-\linewidth \hskip-\@totalleftmargin \hskip\columnwidth}%
 \MakeFramed {\advance\hsize-\width
   \@totalleftmargin\z@ \linewidth\hsize
   \@setminipage}}%
 {\par\unskip\endMakeFramed%
 \at@end@of@kframe}
\makeatother

\definecolor{shadecolor}{rgb}{.97, .97, .97}
\definecolor{messagecolor}{rgb}{0, 0, 0}
\definecolor{warningcolor}{rgb}{1, 0, 1}
\definecolor{errorcolor}{rgb}{1, 0, 0}
\newenvironment{knitrout}{}{} % an empty environment to be redefined in TeX

\usepackage{alltt}
\usepackage[utf8]{inputenc}

\usepackage[american]{babel}
\usepackage{graphicx}
\usepackage{amsmath}
\usepackage{amsthm}
\usepackage{amssymb}
\usepackage{natbib}
\usepackage[margin=0.5in]{geometry}
\usepackage{fancyvrb}
\usepackage{pdfpages}
\usepackage{xcolor,colortbl}

\setlength{\parskip}{.1in}  
\setlength{\parindent}{0.0in}  

\setcounter{tocdepth}{1}
\setcounter{secnumdepth}{1}

\newcommand{\code}[1]{\texttt{#1}}
%opening
\title{Experimental demonstration of Allee effects in microbial populations}
\author{RajReni Kaul, Andrew M. Kramer, Fred Dobbs and John M. Drake}

\usepackage{Sweave}
\IfFileExists{upquote.sty}{\usepackage{upquote}}{}
\begin{document}


\maketitle

\begin{abstract}
Italics (non-scientific) are notes to self. Bold sentences are paragraph purpose. 
\end{abstract}

\section{Introduction}
\textbf{hook without the straw man, We know that microbes are dispersal limited, but there is still the assumption that dispersal into an ideal environment is the major limiting factor for establishing a new population}

Microbiology has been transformed by the molecular renaissance. Molecular techniques allowed exploration of previously intractable questions leading to a conceptual shift of the Becking-Bass(BB) tenet \textit{'Everything is everywhere...'} (\cite{baas_becking_geobiologie_1934, de_wit_everything_2006}). Multiple studies have shown that microbes are dispersal limited, but few studies have examined the second component \textit{'...but the environment selects'} of the BB tenet (\cite{sul_marine_2013, reno_biogeography_2009, hellweger_biogeographic_2014}). The updated tenet implies that a population will establish as long as the cell can disperse to a favorable physiological environment and is akin to the dogmatic view of microbial contamination resulting from the introduction of a single cell.  The molecular renasance has shown that populations of micro- and macro- organisms can be subject to similar dynamical processes, so the single cell assumption may not be reasonable when put into the density dependence framework of macro-organisms.

\textit{Why is it unreasonable to assume that a single asexual organism can propagate a colony? Need a stronger statement to connect to AE?}

\textbf{Allee effects or positive density dependence can have unexpected impact on population survival}

An Allee effect (AE) or positive density dependence is characterized by reduced or negative growth rates at small populations (\cite{allee_studies_1932}) and has been observed in Mollusca, Arthropoda and Chordata (as reviewed in \cite{kramer_evidence_2009}).  Strong Allee effects result in small populations having negative per capita growth rates which result in an extinction vortex. This effect and the minimum population required for a positive growth rate can be detected in the population's probability of establishment when plotted as a function of initial size. Populations not impacted by an Allee effect will have a monotonic increase in the probability of establishment with increasing initial density, demographic stochasticity is driving the relationship.  A strong Allee effect would create a sigmoidal relationship centered around the critical threshold, below which the probability of establish would increase slower than above due to negative per capita growth rates (\cite{dennis_allee_2002}). The presence and strength of an Allee effect is important for management of exploited and vulnerable populations as well as nuisance or invasive species. Allee effects have been observed in natural studies (\cite{angulo_double_2007} predator-induced review \cite{gascoigne_allee_2004}), or experimental manipulations of macro-organisms (\cite{kramer_experimental_2010, ward_predators_2008}). However, the only study to explore the possibility of an AE in microbes used an engineered yeast, designed to have an Allee effect (\cite{dai_generic_2012}).

\textbf{We were our study picks up}

In the current study, we used a combined theoretical and empirical approach to detect the presence of Allee effects in microbial populations of \textit{Vibrio fischeri}, strain \textit{ES114} containing a plasmid with constitutively expressed green florescence protein (GFP) and kanamycin-resistance cassettes (pVSV102, \cite{dunn_new_2006}). \textit{V. fischeri} populations were propagated from a range of inoculum sizes, a subset of which were also exposed to predation by the eukaryotic generalist bacterivore \textit{Cafeteria roenbergensis}. The success or failure of establishment was then used to determine the presence and strength an intrinsic or predator-induced Allee effect based on the relationship between probability of establishment and inoculum size. 

\section{Methods}
\textit{Experiment}

The presence of demographic or predator induced Allee effects were examined using a partial factorial design. Populations of \textit{V. fisheri} were inoculated in high and low resource environments, a portion of the populations with low resources were also exposed to predation by \textit{C. roenbergensis}. Populations of \textit{V. fisheri} with high resources were not exposed to predation due to co-culturing limitations. All populations were grown in mineral salts medium (0.4mM NaPO$_{4}$ (pH 7.5), 50mM Tris (pH 8.0), 11mM NH$_{4}$Cl, 10uM  FeSO$_{4}\cdot$7H$_{2}$O, 55mM MgSO$_{4}\cdot$7H$_{2}$O, 11mM KCl, 0.3M NaCl, 11mM CaCl$_{2}\cdot$2H$_{2}$O) containing 20 mM glycerol under kanamycin selection (100mg/mL). The high resource environment had an additional 20mM glycerol. 

Individual cells of \textit{Vf} and \textit{C. roenbergensis} were selected to inoculate experimental populations by fluorescence flow cytometry. Exponential phase cultures of \textit{Vf} and \textit{C. roenbergensis} were stained with 3uM propidium iodide (PI),a membrane impermeable nucleic acid dye, for 15 minutes prior to sorting into a well pre-filled with media of a microtiter plate. This method allowed populations to be accurately initiated with geometrically increasing number of viable cells (1 to 64 cells in high resource and 1 to 2048 cells in low resource). The high resource experiment was conducted in 3 replicate 96 well microplates with a final volume of 200uL. The low resource experiment was done in a 384 well microplate with a final volume of 75uL. The volume difference created non-identical but overlapping density treatments of 5, 10, 20, 40, 80, 160 and 320 cells/mL for the high resource treatment (n=36) and  13.3, 26.7, 53.3, 106.7, 213.3, 426.7, 853.3, 1706.7, 3413.3, 6826.7, 13653.3, and 27306.7 cells/mL for the low resource treatment (n=24). A predator-induced AE was tested by adding 10 \textit{C. roenbergensis} per population to half of the low resource treatment replicates.  

Immediately following well inoculation the microtiter plates were sealed with optical film to avoid any contamination. Plates were simultaneously incubated at 28C and monitored for growth based on optical density (OD$_{620}$) and GFP fluorescence (485/528, sensitivity=XX) in a Synergy H1 plate reader (Biotek). Populations were allowed to grow for 72 hours before assessing population establishment. Established populations, scored as 1, were defined as having an increase of at least 0.25 OD$_{620}$ or 100 relative fluorescence units (RFUs) from the initial reading. The threshold used ensured that noise introduced by the plate reader would not be considered growth. \textit{V. fisheri} strain ES114  and \textit{C. roenbergensis} were kindly supplied by Eric Stabb (UGA) and Alexander Bochdansky(), respectively.  
  
\textit{Model Fitting}

The presence of an AE can be separated from stochastic noise by examining the probability of establishment as a function of initial population density. The probability of small populations establishing when regulated by an Allee effect will be due to a negative per capita growth rate and demographic stochasticity. When a population with an Allee effect is inoculated with more cells than the critical density, probability of establishment will only be reduced due to demographic stochasticity. The different ranges that a population experiences negative per capita growth rate and demographic stochasticity creates a sigmoid when probability of establishment is a function of initial population size. The 2-parameter Weibull function is sigmoidal when the shape parameter is greater than one ($k>1$). Using this as a statistical test for the presence of an AE, we simultaneously fit the shape ($k$) and scale ($\lambda$) parameters using MLE based on a binomial distribution of establishment against natural log transformed density. Confidence intervals around the point estimates assume a normal distribution. 


\begin{figure}
\[ y = 1-e^{(\frac{-x}{\lambda})^{k}} \]
\caption{\textbf{Expression used to detect AE} The probability of persistence, $y$,is a function of the population density, $x$. An estimate of $k>1$ indicates the presence of an AE. \textit{Need to determine a way to embed equation without calling it a figure.}}
\end{figure}

\section{Results}

Roughly 90\% of the \textit{V. fischeri} culture met the viable cell criteria set by the flow cytometer. The yield or carrying capacity of a population was not effected by initial density within the treatments. The shape parameter,$k$, for all three treatments was estimated to be greater than 1. The scale parameter,$\lambda$, for the three environments was also greater than 0 indicating the density needed for a positive growth rate is larger than 1 $cell mL^{-1}$ (Fig.1) The \textit{V. fischeri} populations in this experimental system are subject to strong Allee effects. 
   
\section{Discussion}

\textbf{Findings and applications}

This work adds to the growing knowledge of AE impact on natural populations. \textit{V. fischeri} are subject to both intrinsic and predator included Allee effects. The strength of the effect, represented by the critical density, increases with decreasing environmental quality (Fig. 2). We would expect even more pronounced effects in natural marine populations since dissolved organic carbon (DOC) are three orders of magnitude lower (\cite{pedler_single_2014}).  Understanding this relationship under climate change mediated DOC shifts in marine systems will increase our understanding of heterotrophic microbes in the global carbon cycle \textit{Will this increase or decrease sink strength of aquatic systems?}.  Understanding the impacts of Allee effects extends far beyond marine symbionts to the spread of all microbes, determining infectious dose and even the growth of tumors() \cite{regoes_dose-dependent_2002,litchman_invisible_2010, korolev_turning_2014}, respectively).
	
\textbf{Objections}

Our methods did not allow us to differentiate between extremely slow growth and extinction, but both result in a 'failure to establish'. This study detected the presence of an AE, but not the mechanisms leading to the critical density. A candidate mechanism is quorum sensing, which detects density, and is known to be important in \textit{V. fischeri} forming symbiosis with bobtail squid \textit{Euprymna scolopes}. Many other species have similar interactions based on population density (\cite{waters_quorum_2005,de_kievit_bacterial_2000, gascoigne_allee_2004}cites quorum sensing; Pathogens quorum sensing reviewed; cite examples of macro organisms with AE behavior mechanism). 
  
\textbf{Call to integrate microbial and ecological knowledge to understand microbial populations}

Our conceptual understanding of mechanisms controlling microbial populations is much more complex than the original \textit{`Everything is everywhere`} tenet (\cite{baas_becking_geobiologie_1934}). Microbial ecology has shown that many mechanisms that control macro-organisms, also influence microorganisms (\cite{reno_biogeography_2009, hellweger_biogeographic_2014}, as reviewed in \cite{prosser_role_2007}). This study continues the trend by presenting an example of density dependence of a microbial population in a rigorous theoretical framework. Work of this nature could be used to further develop long standing microbial concepts such as quorum protection (\cite{macreadie_quorum_2015}) and minimum inhibitory concentration (\cite{steels_sorbic_2000, galgiani_turbidimetric_1978}). Furthermore, this work integrates microbial and ecological knowledge to create a highly manipulatable experimental system allowing gains in both fields.  


\section{References}
\bibliographystyle{plainnat}%Choose a bibliograhpic style
\bibliography{MS_citation}
\section{Figures}












\begin{figure}[!h]
\begin{center}
\begin{knitrout}
\definecolor{shadecolor}{rgb}{0.969, 0.969, 0.969}\color{fgcolor}
\includegraphics[width=\maxwidth]{figure/fig1-1} 

\end{knitrout}
\end{center}
\caption{\textbf{Parameter estimates suggest Allee Effect present in all treatments.} The $k$ parameter is used to test the presence of an AE. Values greater than 1 indicate a sigmoidal relationship between density and probability of establishment. The shape parameter, $\lambda$, is the upper bound on the inflection point or critical density required to escape the AE. The point parameter estimates are presented with 95\% confidence intervals.  \textit{Include explanation of treatments?}
}
\end{figure}

\begin{figure}
\begin{center}
\begin{knitrout}
\definecolor{shadecolor}{rgb}{0.969, 0.969, 0.969}\color{fgcolor}
\includegraphics[width=\maxwidth]{figure/fig2-1} 

\end{knitrout}
\end{center}
\caption{\textbf{Weibull curve based on estimated parameters.} Point is based on parameter point estimate, the shaded area is based on the 95\% confidence interval. The values are: X-X for the high resource (40mM glycerol), Y-Y low resource (20mM glycerol) and Z-Z low resource with predation (20mM glycerol plus 134 Cafeteria/mL)  
}
\end{figure}

\end{document}
